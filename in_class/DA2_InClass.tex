\documentclass[]{article}
\usepackage{lmodern}
\usepackage{amssymb,amsmath}
\usepackage{ifxetex,ifluatex}
\usepackage{fixltx2e} % provides \textsubscript
\ifnum 0\ifxetex 1\fi\ifluatex 1\fi=0 % if pdftex
  \usepackage[T1]{fontenc}
  \usepackage[utf8]{inputenc}
\else % if luatex or xelatex
  \ifxetex
    \usepackage{mathspec}
  \else
    \usepackage{fontspec}
  \fi
  \defaultfontfeatures{Ligatures=TeX,Scale=MatchLowercase}
\fi
% use upquote if available, for straight quotes in verbatim environments
\IfFileExists{upquote.sty}{\usepackage{upquote}}{}
% use microtype if available
\IfFileExists{microtype.sty}{%
\usepackage{microtype}
\UseMicrotypeSet[protrusion]{basicmath} % disable protrusion for tt fonts
}{}
\usepackage[margin=1in]{geometry}
\usepackage{hyperref}
\hypersetup{unicode=true,
            pdftitle={Lab 2: Summarizing Multivariate Data},
            pdfauthor={Anish Yakkala},
            pdfborder={0 0 0},
            breaklinks=true}
\urlstyle{same}  % don't use monospace font for urls
\usepackage{color}
\usepackage{fancyvrb}
\newcommand{\VerbBar}{|}
\newcommand{\VERB}{\Verb[commandchars=\\\{\}]}
\DefineVerbatimEnvironment{Highlighting}{Verbatim}{commandchars=\\\{\}}
% Add ',fontsize=\small' for more characters per line
\usepackage{framed}
\definecolor{shadecolor}{RGB}{248,248,248}
\newenvironment{Shaded}{\begin{snugshade}}{\end{snugshade}}
\newcommand{\AlertTok}[1]{\textcolor[rgb]{0.94,0.16,0.16}{#1}}
\newcommand{\AnnotationTok}[1]{\textcolor[rgb]{0.56,0.35,0.01}{\textbf{\textit{#1}}}}
\newcommand{\AttributeTok}[1]{\textcolor[rgb]{0.77,0.63,0.00}{#1}}
\newcommand{\BaseNTok}[1]{\textcolor[rgb]{0.00,0.00,0.81}{#1}}
\newcommand{\BuiltInTok}[1]{#1}
\newcommand{\CharTok}[1]{\textcolor[rgb]{0.31,0.60,0.02}{#1}}
\newcommand{\CommentTok}[1]{\textcolor[rgb]{0.56,0.35,0.01}{\textit{#1}}}
\newcommand{\CommentVarTok}[1]{\textcolor[rgb]{0.56,0.35,0.01}{\textbf{\textit{#1}}}}
\newcommand{\ConstantTok}[1]{\textcolor[rgb]{0.00,0.00,0.00}{#1}}
\newcommand{\ControlFlowTok}[1]{\textcolor[rgb]{0.13,0.29,0.53}{\textbf{#1}}}
\newcommand{\DataTypeTok}[1]{\textcolor[rgb]{0.13,0.29,0.53}{#1}}
\newcommand{\DecValTok}[1]{\textcolor[rgb]{0.00,0.00,0.81}{#1}}
\newcommand{\DocumentationTok}[1]{\textcolor[rgb]{0.56,0.35,0.01}{\textbf{\textit{#1}}}}
\newcommand{\ErrorTok}[1]{\textcolor[rgb]{0.64,0.00,0.00}{\textbf{#1}}}
\newcommand{\ExtensionTok}[1]{#1}
\newcommand{\FloatTok}[1]{\textcolor[rgb]{0.00,0.00,0.81}{#1}}
\newcommand{\FunctionTok}[1]{\textcolor[rgb]{0.00,0.00,0.00}{#1}}
\newcommand{\ImportTok}[1]{#1}
\newcommand{\InformationTok}[1]{\textcolor[rgb]{0.56,0.35,0.01}{\textbf{\textit{#1}}}}
\newcommand{\KeywordTok}[1]{\textcolor[rgb]{0.13,0.29,0.53}{\textbf{#1}}}
\newcommand{\NormalTok}[1]{#1}
\newcommand{\OperatorTok}[1]{\textcolor[rgb]{0.81,0.36,0.00}{\textbf{#1}}}
\newcommand{\OtherTok}[1]{\textcolor[rgb]{0.56,0.35,0.01}{#1}}
\newcommand{\PreprocessorTok}[1]{\textcolor[rgb]{0.56,0.35,0.01}{\textit{#1}}}
\newcommand{\RegionMarkerTok}[1]{#1}
\newcommand{\SpecialCharTok}[1]{\textcolor[rgb]{0.00,0.00,0.00}{#1}}
\newcommand{\SpecialStringTok}[1]{\textcolor[rgb]{0.31,0.60,0.02}{#1}}
\newcommand{\StringTok}[1]{\textcolor[rgb]{0.31,0.60,0.02}{#1}}
\newcommand{\VariableTok}[1]{\textcolor[rgb]{0.00,0.00,0.00}{#1}}
\newcommand{\VerbatimStringTok}[1]{\textcolor[rgb]{0.31,0.60,0.02}{#1}}
\newcommand{\WarningTok}[1]{\textcolor[rgb]{0.56,0.35,0.01}{\textbf{\textit{#1}}}}
\usepackage{graphicx,grffile}
\makeatletter
\def\maxwidth{\ifdim\Gin@nat@width>\linewidth\linewidth\else\Gin@nat@width\fi}
\def\maxheight{\ifdim\Gin@nat@height>\textheight\textheight\else\Gin@nat@height\fi}
\makeatother
% Scale images if necessary, so that they will not overflow the page
% margins by default, and it is still possible to overwrite the defaults
% using explicit options in \includegraphics[width, height, ...]{}
\setkeys{Gin}{width=\maxwidth,height=\maxheight,keepaspectratio}
\IfFileExists{parskip.sty}{%
\usepackage{parskip}
}{% else
\setlength{\parindent}{0pt}
\setlength{\parskip}{6pt plus 2pt minus 1pt}
}
\setlength{\emergencystretch}{3em}  % prevent overfull lines
\providecommand{\tightlist}{%
  \setlength{\itemsep}{0pt}\setlength{\parskip}{0pt}}
\setcounter{secnumdepth}{0}
% Redefines (sub)paragraphs to behave more like sections
\ifx\paragraph\undefined\else
\let\oldparagraph\paragraph
\renewcommand{\paragraph}[1]{\oldparagraph{#1}\mbox{}}
\fi
\ifx\subparagraph\undefined\else
\let\oldsubparagraph\subparagraph
\renewcommand{\subparagraph}[1]{\oldsubparagraph{#1}\mbox{}}
\fi

%%% Use protect on footnotes to avoid problems with footnotes in titles
\let\rmarkdownfootnote\footnote%
\def\footnote{\protect\rmarkdownfootnote}

%%% Change title format to be more compact
\usepackage{titling}

% Create subtitle command for use in maketitle
\newcommand{\subtitle}[1]{
  \posttitle{
    \begin{center}\large#1\end{center}
    }
}

\setlength{\droptitle}{-2em}

  \title{Lab 2: Summarizing Multivariate Data}
    \pretitle{\vspace{\droptitle}\centering\huge}
  \posttitle{\par}
    \author{Anish Yakkala}
    \preauthor{\centering\large\emph}
  \postauthor{\par}
    \date{}
    \predate{}\postdate{}
  

\begin{document}
\maketitle

\hypertarget{reading-the-data}{%
\subsection{Reading the Data}\label{reading-the-data}}

In this lab, we will examine data college tuition rates, based on a 1995
dataset from US News and World Report.

\begin{Shaded}
\begin{Highlighting}[]
\CommentTok{# Read Data}
\NormalTok{colleges =}\StringTok{ }\KeywordTok{read.csv}\NormalTok{(}\StringTok{'http://kbodwin.web.unc.edu/files/2016/09/tuition_final.csv'}\NormalTok{)}

\CommentTok{# Adjust labels for later}

\NormalTok{colleges <-}\StringTok{ }\NormalTok{colleges }\OperatorTok\StringTok{ }\KeywordTok{mutate}\NormalTok{(}
  \DataTypeTok{Name =} \KeywordTok{gsub}\NormalTok{(}\StringTok{"California State Univ. at"}\NormalTok{, }\StringTok{"CSU"}\NormalTok{, Name),}
  \DataTypeTok{Name =} \KeywordTok{gsub}\NormalTok{(}\StringTok{"California State University at"}\NormalTok{, }\StringTok{"CSU"}\NormalTok{, Name),}
  \DataTypeTok{Name =} \KeywordTok{gsub}\NormalTok{(}\StringTok{"California Polytechnic"}\NormalTok{, }\StringTok{"Cal Poly"}\NormalTok{, Name),}
  \DataTypeTok{Name =} \KeywordTok{gsub}\NormalTok{(}\StringTok{"California Poly"}\NormalTok{, }\StringTok{"Cal Poly"}\NormalTok{, Name),}
  \DataTypeTok{Name =} \KeywordTok{gsub}\NormalTok{(}\StringTok{"University of California at"}\NormalTok{, }\StringTok{"UC"}\NormalTok{, Name)}
\NormalTok{)}
\end{Highlighting}
\end{Shaded}

Check out the \texttt{summary()} of the dataset and familiarize yourself
with the variables.

\begin{Shaded}
\begin{Highlighting}[]
\KeywordTok{summary}\NormalTok{(colleges)}
\end{Highlighting}
\end{Shaded}

\begin{verbatim}
##        ID            Name               State         Public    
##  Min.   : 1002   Length:1302        NY     :101   Min.   :1.00  
##  1st Qu.: 1874   Class :character   PA     : 83   1st Qu.:1.00  
##  Median : 2650   Mode  :character   CA     : 70   Median :2.00  
##  Mean   : 3126                      TX     : 60   Mean   :1.64  
##  3rd Qu.: 3431                      MA     : 56   3rd Qu.:2.00  
##  Max.   :30431                      OH     : 52   Max.   :2.00  
##                                     (Other):880                 
##     Avg.SAT        Avg.ACT       Applied         Accepted    
##  Min.   : 600   Min.   :11    Min.   :   35   Min.   :   35  
##  1st Qu.: 884   1st Qu.:20    1st Qu.:  696   1st Qu.:  554  
##  Median : 957   Median :22    Median : 1470   Median : 1095  
##  Mean   : 968   Mean   :22    Mean   : 2752   Mean   : 1871  
##  3rd Qu.:1038   3rd Qu.:24    3rd Qu.: 3314   3rd Qu.: 2303  
##  Max.   :1410   Max.   :31    Max.   :48094   Max.   :26330  
##  NA's   :523    NA's   :588   NA's   :10      NA's   :11     
##       Size        Out.Tuition       Spending    
##  Min.   :   59   Min.   : 1044   Min.   : 1834  
##  1st Qu.:  966   1st Qu.: 6111   1st Qu.: 6116  
##  Median : 1812   Median : 8670   Median : 7729  
##  Mean   : 3693   Mean   : 9277   Mean   : 8988  
##  3rd Qu.: 4540   3rd Qu.:11659   3rd Qu.:10054  
##  Max.   :31643   Max.   :25750   Max.   :62469  
##  NA's   :3       NA's   :20      NA's   :39
\end{verbatim}

\begin{Shaded}
\begin{Highlighting}[]
\KeywordTok{skim}\NormalTok{(colleges)}
\end{Highlighting}
\end{Shaded}

\begin{verbatim}
## Skim summary statistics
##  n obs: 1302 
##  n variables: 11 
## 
## -- Variable type:character -----------------------------------------------------
##  variable missing complete    n min max empty n_unique
##      Name       0     1302 1302   8  45     0     1274
## 
## -- Variable type:factor --------------------------------------------------------
##  variable missing complete    n n_unique                      top_counts
##     State       0     1302 1302       51 NY: 101, PA: 83, CA: 70, TX: 60
##  ordered
##    FALSE
## 
## -- Variable type:integer -------------------------------------------------------
##     variable missing complete    n    mean      sd   p0     p25    p50
##     Accepted      11     1291 1302 1870.68 2250.87   35  554.5  1095  
##      Applied      10     1292 1302 2752.1  3541.97   35  695.75 1470  
##      Avg.ACT     588      714 1302   22.12    2.58   11   20.25   22  
##      Avg.SAT     523      779 1302  967.98  123.58  600  884.5   957  
##           ID       0     1302 1302 3126.37 2970.17 1002 1874.5  2649.5
##  Out.Tuition      20     1282 1302 9276.91 4170.77 1044 6111    8670  
##       Public       0     1302 1302    1.64    0.48    1    1       2  
##         Size       3     1299 1302 3692.67 4544.85   59  966    1812  
##     Spending      39     1263 1302 8987.89 5347.46 1834 6115.5  7729  
##       p75  p100     hist
##   2303    26330 ▇▁▁▁▁▁▁▁
##   3314.25 48094 ▇▁▁▁▁▁▁▁
##     24       31 ▁▁▁▇▇▅▁▁
##   1038     1410 ▁▁▆▇▅▂▁▁
##   3430.75 30431 ▇▁▁▁▁▁▁▁
##  11659    25750 ▂▇▇▅▂▂▁▁
##      2        2 ▅▁▁▁▁▁▁▇
##   4539.5  31643 ▇▂▁▁▁▁▁▁
##  10054    62469 ▇▃▁▁▁▁▁▁
\end{verbatim}

\begin{itemize}
\tightlist
\item
  Briefly describe the variables in this dataset - what do they
  represent? What are the possible values?
\end{itemize}

List of colleges with a bunch of features about them, and ID is the id
of the school. *** \#\# Adjusting and cleaning the data

Note that the variable ``Public'' tells us whether a school is public or
private. However, the categories are labeled ``1'' and ``2'', so R
thinks this is a numerical value. We will use the dplyr function
\texttt{mutate()} to adjust this variable, and also to create a new one
called \texttt{Acc.Rate}.

\begin{Shaded}
\begin{Highlighting}[]
\NormalTok{colleges <-}\StringTok{ }\NormalTok{colleges }\OperatorTok\StringTok{ }
\StringTok{  }\KeywordTok{mutate}\NormalTok{(}
    \DataTypeTok{Public =} \KeywordTok{factor}\NormalTok{(Public, }\DataTypeTok{labels =} \KeywordTok{c}\NormalTok{(}\StringTok{"Public"}\NormalTok{, }\StringTok{"Private"}\NormalTok{)),}
    \DataTypeTok{Acc.Rate =}\NormalTok{ Accepted}\OperatorTok{/}\NormalTok{Applied}
\NormalTok{  )}
\end{Highlighting}
\end{Shaded}

\begin{itemize}
\tightlist
\item
  What information is now contained in the variable Acc.Rate? Why might
  we prefer this to the original data?
\end{itemize}

Proportion of students that were accepted out of applied. This gives us
what we really want to know from the two variables, accepted and
applied. Also gives us a good way to compare the schools in a
standardized way.

\begin{itemize}
\tightlist
\item
  What would happen if we removed ``colleges \textless{}-'' from the
  beginning of the above code?
\end{itemize}

We would not save this change into the colleges dataset.

\begin{center}\rule{0.5\linewidth}{\linethickness}\end{center}

Now we would like to take a look at the public schools in California,
and compare them using our multivariate plotting tools. First, let's
make a dataset containing only these schools using the dplyr function
\texttt{filter()}.

\begin{Shaded}
\begin{Highlighting}[]
\CommentTok{# Dataset of only CA public schools}
\NormalTok{CA_public <-}\StringTok{ }\NormalTok{colleges }\OperatorTok
\StringTok{  }\KeywordTok{filter}\NormalTok{(State }\OperatorTok{==}\StringTok{ "CA"}\NormalTok{, Public }\OperatorTok{==}\StringTok{ "Public"}\NormalTok{) }\OperatorTok\StringTok{ }
\StringTok{  }\KeywordTok{select}\NormalTok{(Name, Avg.SAT, Acc.Rate, Size, Out.Tuition, Spending) }\OperatorTok
\StringTok{  }\KeywordTok{na.omit}\NormalTok{()}
\end{Highlighting}
\end{Shaded}

\begin{itemize}
\tightlist
\item
  What did we do in the third line of code, using ``select()''?
\end{itemize}

It selectes the columns specified in the argument

\begin{itemize}
\tightlist
\item
  What did ``na.omit()'' do in the fourth line of code?
\end{itemize}

Removes any rows with nans in them.

\begin{center}\rule{0.5\linewidth}{\linethickness}\end{center}

\hypertarget{face-plots}{%
\subsection{Face Plots}\label{face-plots}}

Now we can make plots for the CA public schools for which we have full
information. We will start with the Chernoff faces plotting style.

\begin{Shaded}
\begin{Highlighting}[]
\NormalTok{CA_public }\OperatorTok
\StringTok{  }\KeywordTok{select}\NormalTok{(Avg.SAT, Acc.Rate, Size, Out.Tuition, Spending) }\OperatorTok
\StringTok{  }\KeywordTok{faces}\NormalTok{(}\DataTypeTok{labels =}\NormalTok{ CA_public}\OperatorTok{$}\NormalTok{Name)}
\end{Highlighting}
\end{Shaded}

\includegraphics{DA2_InClass_files/figure-latex/unnamed-chunk-6-1.pdf}

\begin{verbatim}
## effect of variables:
##  modified item       Var          
##  "height of face   " "Avg.SAT"    
##  "width of face    " "Acc.Rate"   
##  "structure of face" "Size"       
##  "height of mouth  " "Out.Tuition"
##  "width of mouth   " "Spending"   
##  "smiling          " "Avg.SAT"    
##  "height of eyes   " "Acc.Rate"   
##  "width of eyes    " "Size"       
##  "height of hair   " "Out.Tuition"
##  "width of hair   "  "Spending"   
##  "style of hair   "  "Avg.SAT"    
##  "height of nose  "  "Acc.Rate"   
##  "width of nose   "  "Size"       
##  "width of ear    "  "Out.Tuition"
##  "height of ear   "  "Spending"
\end{verbatim}

\begin{itemize}
\tightlist
\item
  What do you observe from this plot? Which schools are similar to each
  other? Which schools stand out as unique?
\end{itemize}

Fullerton, SJSU, CSU Northridge and SFSU seem similar. Also UC Irvine
and UC San Diego. Sonomo State and Westmont College seem pretty unique.

\begin{itemize}
\tightlist
\item
  What are the advantages and disadvantages of the face plot style?
\end{itemize}

Pretty easy to notice differences and compare. However once there is a
lot of features it can become hard to track all of them. Also if this is
taking into account bigger values with a happier face, then it can be
innapropriate for some cases.

\begin{center}\rule{0.5\linewidth}{\linethickness}\end{center}

\hypertarget{star-plots}{%
\subsection{Star plots}\label{star-plots}}

One downside of the face plots is that it is difficult to tell which
specific variables are being compared. A solution to this is called
``star plots''.

\begin{Shaded}
\begin{Highlighting}[]
\NormalTok{CA_public }\OperatorTok
\StringTok{  }\KeywordTok{select}\NormalTok{(Avg.SAT, Acc.Rate, Size, Out.Tuition, Spending) }\OperatorTok
\StringTok{  }\KeywordTok{stars}\NormalTok{(}\DataTypeTok{labels =}\NormalTok{ CA_public}\OperatorTok{$}\NormalTok{Name, }\DataTypeTok{nrow =} \DecValTok{4}\NormalTok{, }\DataTypeTok{key.loc =} \KeywordTok{c}\NormalTok{(}\DecValTok{0}\NormalTok{,}\DecValTok{10}\NormalTok{), }\DataTypeTok{cex =} \FloatTok{.5}\NormalTok{)}
\end{Highlighting}
\end{Shaded}

\includegraphics{DA2_InClass_files/figure-latex/unnamed-chunk-7-1.pdf}

\begin{itemize}
\tightlist
\item
  Which school has the highest out-of-state tuition in this data?
\end{itemize}

Westmont College

\begin{itemize}
\tightlist
\item
  Which school has the lowest acceptance rate in this data?
\end{itemize}

Sonoma State

\begin{itemize}
\tightlist
\item
  Which two schools seem most similar to you, based on these plots?
\end{itemize}

SJSU and Fullerton

\begin{itemize}
\tightlist
\item
  What are the advantages and disadvantages of this plot style?
\end{itemize}

A positive is it that it is a good way to merge all variables in a easy
way to compare between groups. Geometrically it can be hard to tell
which variable is being extended.

\begin{center}\rule{0.5\linewidth}{\linethickness}\end{center}

\hypertarget{segmented-star-plots}{%
\subsection{Segmented Star Plots}\label{segmented-star-plots}}

To better visualize our data, we can ``segment'' our star plots, and
plot them as slices of a circular area rather than as lines in a
polygon.

\begin{Shaded}
\begin{Highlighting}[]
\NormalTok{CA_public }\OperatorTok
\StringTok{  }\KeywordTok{select}\NormalTok{(Avg.SAT, Acc.Rate, Size, Out.Tuition, Spending) }\OperatorTok
\StringTok{  }\KeywordTok{stars}\NormalTok{(}\DataTypeTok{labels =}\NormalTok{ CA_public}\OperatorTok{$}\NormalTok{Name, }\DataTypeTok{nrow =} \DecValTok{4}\NormalTok{, }\DataTypeTok{key.loc =} \KeywordTok{c}\NormalTok{(}\DecValTok{0}\NormalTok{,}\DecValTok{10}\NormalTok{), }\DataTypeTok{cex =} \FloatTok{.5}\NormalTok{, }\DataTypeTok{draw.segments =} \OtherTok{TRUE}\NormalTok{, }\DataTypeTok{col.segments =} \KeywordTok{rainbow}\NormalTok{(}\DecValTok{6}\NormalTok{))}
\end{Highlighting}
\end{Shaded}

\includegraphics{DA2_InClass_files/figure-latex/unnamed-chunk-8-1.pdf}

\begin{itemize}
\tightlist
\item
  In what way does Cal Poly most differ from the other CSUs?
\end{itemize}

Acceptance Rate and Avg Sat

\begin{itemize}
\tightlist
\item
  In what way does Westmont College most differ from the other schools?
\end{itemize}

Out of State Tuition

\begin{itemize}
\tightlist
\item
  Which is the largest public school in California?
\end{itemize}

UCLA

\begin{itemize}
\tightlist
\item
  What are the advantages and disadvantages of this plot style?
\end{itemize}

I really like that they color code this because it makes it easier to
see which parts they are talking about. When things get small it is hard
to compare two small ones.

\begin{center}\rule{0.5\linewidth}{\linethickness}\end{center}

\hypertarget{scatterplots}{%
\subsection{Scatterplots}\label{scatterplots}}

Finally, we can think about pairwise relationships between variables,
rather than comparisons between schools. We will use the
\texttt{ggplot2} framework for these plots. Make sure that you
understand each step in the code below:

\begin{Shaded}
\begin{Highlighting}[]
\NormalTok{colleges }\OperatorTok
\StringTok{  }\KeywordTok{ggplot}\NormalTok{(}\KeywordTok{aes}\NormalTok{(}\DataTypeTok{x =}\NormalTok{ Avg.SAT, }\DataTypeTok{y =}\NormalTok{ Out.Tuition)) }\OperatorTok{+}\StringTok{ }\KeywordTok{geom_point}\NormalTok{() }\OperatorTok{+}\StringTok{ }\KeywordTok{facet_grid}\NormalTok{(}\OperatorTok{~}\NormalTok{Public)}
\end{Highlighting}
\end{Shaded}

\begin{verbatim}
## Warning: Removed 535 rows containing missing values (geom_point).
\end{verbatim}

\includegraphics{DA2_InClass_files/figure-latex/unnamed-chunk-9-1.pdf}

\begin{itemize}
\tightlist
\item
  Based on these scatterplots, is there a relationship between Average
  SAT score and the out-of-state tuition of a school?
\end{itemize}

Seems like there is a positive association.

\begin{itemize}
\tightlist
\item
  Based on these scatterplots, is the relationship different for Public
  versus Private schools?
\end{itemize}

Does not appear so

\begin{center}\rule{0.5\linewidth}{\linethickness}\end{center}

We can also quite easily make all the possible scatterplots for the
dataset using the \texttt{ggplot} function \texttt{ggpairs}:

\begin{Shaded}
\begin{Highlighting}[]
\NormalTok{colleges }\OperatorTok
\StringTok{  }\KeywordTok{select}\NormalTok{(Public, Avg.SAT, Acc.Rate, Size, Out.Tuition, Spending) }\OperatorTok
\StringTok{  }\KeywordTok{na.omit}\NormalTok{() }\OperatorTok
\StringTok{  }\KeywordTok{ggpairs}\NormalTok{()}
\end{Highlighting}
\end{Shaded}

\begin{verbatim}
## `stat_bin()` using `bins = 30`. Pick better value with `binwidth`.
## `stat_bin()` using `bins = 30`. Pick better value with `binwidth`.
## `stat_bin()` using `bins = 30`. Pick better value with `binwidth`.
## `stat_bin()` using `bins = 30`. Pick better value with `binwidth`.
## `stat_bin()` using `bins = 30`. Pick better value with `binwidth`.
\end{verbatim}

\includegraphics{DA2_InClass_files/figure-latex/unnamed-chunk-10-1.pdf}

\begin{itemize}
\tightlist
\item
  Are there more public schools or more private schools in this dataset?
\end{itemize}

More Private

\begin{itemize}
\tightlist
\item
  Which variables are very different between public schools and private
  schools? Does this surprise you?
\end{itemize}

Size

\begin{itemize}
\tightlist
\item
  Which variables seem to be strongly positively correlated? In real
  world terms, why does this make sense?
\end{itemize}

Out Tuition and Spending. This makes sense since Out Tuition is a lot
more and that gives them more money to spend.

\begin{itemize}
\tightlist
\item
  Which variables seem to be strongly negatively correlated? In real
  world terms, why does this make sense?
\end{itemize}

Avg Sat and Acceptance Rate. Since schools with a lower acceptance rate
are more prestigious and accept students with a higher Avg Sat score.

\begin{itemize}
\tightlist
\item
  Which variables, if any, seem like they might be Normally distributed?
\end{itemize}

Avg Sat and Out Tuition.


\end{document}
